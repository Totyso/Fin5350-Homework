\documentclass[11pt,]{article}
\usepackage[left=1in,top=1in,right=1in,bottom=1in]{geometry}
\newcommand*{\authorfont}{\fontfamily{phv}\selectfont}
\usepackage[]{mathpazo}


  \usepackage[T1]{fontenc}
  \usepackage[utf8]{inputenc}



\usepackage{abstract}
\renewcommand{\abstractname}{}    % clear the title
\renewcommand{\absnamepos}{empty} % originally center

\renewenvironment{abstract}
 {{%
    \setlength{\leftmargin}{0mm}
    \setlength{\rightmargin}{\leftmargin}%
  }%
  \relax}
 {\endlist}

\makeatletter
\def\@maketitle{%
  \newpage
%  \null
%  \vskip 2em%
%  \begin{center}%
  \let \footnote \thanks
    {\fontsize{18}{20}\selectfont\raggedright  \setlength{\parindent}{0pt} \@title \par}%
}
%\fi
\makeatother




\setcounter{secnumdepth}{0}



\title{What a Trader Should Do  }



\author{\Large Tyson Clark\vspace{0.05in} \newline\normalsize\emph{Utah State University}  }


\date{}

\usepackage{titlesec}

\titleformat*{\section}{\normalsize\bfseries}
\titleformat*{\subsection}{\normalsize\itshape}
\titleformat*{\subsubsection}{\normalsize\itshape}
\titleformat*{\paragraph}{\normalsize\itshape}
\titleformat*{\subparagraph}{\normalsize\itshape}


\usepackage{natbib}
\bibliographystyle{apsr}



\newtheorem{hypothesis}{Hypothesis}
\usepackage{setspace}

\makeatletter
\@ifpackageloaded{hyperref}{}{%
\ifxetex
  \usepackage[setpagesize=false, % page size defined by xetex
              unicode=false, % unicode breaks when used with xetex
              xetex]{hyperref}
\else
  \usepackage[unicode=true]{hyperref}
\fi
}
\@ifpackageloaded{color}{
    \PassOptionsToPackage{usenames,dvipsnames}{color}
}{%
    \usepackage[usenames,dvipsnames]{color}
}
\makeatother
\hypersetup{breaklinks=true,
            bookmarks=true,
            pdfauthor={Tyson Clark (Utah State University)},
             pdfkeywords = {Markets, Efficiency, Trading, Algorithmic Trading},  
            pdftitle={What a Trader Should Do},
            colorlinks=true,
            citecolor=blue,
            urlcolor=blue,
            linkcolor=magenta,
            pdfborder={0 0 0}}
\urlstyle{same}  % don't use monospace font for urls



\begin{document}
	
% \pagenumbering{arabic}% resets `page` counter to 1 
%
% \maketitle

{% \usefont{T1}{pnc}{m}{n}
\setlength{\parindent}{0pt}
\thispagestyle{plain}
{\fontsize{18}{20}\selectfont\raggedright 
\maketitle  % title \par  

}

{
   \vskip 13.5pt\relax \normalsize\fontsize{11}{12} 
\textbf{\authorfont Tyson Clark} \hskip 15pt \emph{\small Utah State University}   

}

}







\begin{abstract}

    \hbox{\vrule height .2pt width 39.14pc}

    \vskip 8.5pt % \small 

\noindent The stock market is a great method for any individual, group of
individuals, or entity to participate in the growth of any particular
company. I would like to argue that today's society has turned markets
into a race for a quick buck, more so than ever, with algorithmic
trading. I believe that markets would be better off if investors held
securities for the long-term, along with some market-makers to provide
liquidity, but high-frequency traders have changed our views on the
markets as a whole.


\vskip 8.5pt \noindent \emph{Keywords}: Markets, Efficiency, Trading, Algorithmic Trading \par

    \hbox{\vrule height .2pt width 39.14pc}



\end{abstract}


\vskip 6.5pt

\noindent \doublespacing \begin{quote}
\end{quote}

\section{Buchanan}\label{buchanan}

\subsection{Buchanan}\label{buchanan-1}

Buchanan begins by disputing a seemingly simple phrase by Jacob Viner,
that `economics is what economists do, and economists are those who do
economics' \citet{Buchanan1979}. He writes about how the classic form of
what an economics is has been adapted to be methods of resource
allocation, but he disregards this definition, and argues that it should
be defined by trade and exchange.

Buchanan tells of Robert Crusoe and Friday. The story goes that Crusoe
was on an island, and uses the resources on the island, and one day
Friday arrives. He explains that in his definition of economics would be
such that no such thing occured until Friday arrived, because that
opened the opportunity for exchange.

Buchanan disputes that this classic view that most economists have about
economics, can't be economics, because it can be looked at as simple
mathematics. If you look at human choice and behavior without exchange,
you can find the result with a simple utility maximization function.
Though computation is necessary in economics, economics has to include
exchange.

\subsection{Catallactics}\label{catallactics}

Buchanan briefly mentions catallactics, an approach to economics that
dates back long before Buchanan, but has been absent from the minds of
many modern economics. When talking of catallactics, he is speaking of
free-markets, and looking at prices as they are, and not as they `should
be'.

\subsection{What do Economists do?}\label{what-do-economists-do}

Buchanan writes of how many economists try to determine the economic
problem as a problem of allocation. He mentions Lord Robbins as one such
economist, and writes of the necessity for making allocative decisions.
This focuses more on a problem, rather than human activity, and can be
solved by a mathematician, and probably better so. He briefly writes of
a classic student studying economics telling the difference between an
economic problem and a technological problem, using an example as
defined in textbooks. He then states that these two problems really have
no difference, since as stated before, either can be solved with a
utility maximization function. This shows how many economists overlook
the catallactic views mentioned by Buchanan.

\subsection{What Should Economists do?}\label{what-should-economists-do}

Buchanan argues that economists should know their subject matter, if
nothing else. He writes that they should be `market-economists'. They
should focus on exchange, trade, and human behavior effecting these
things. He says that they should stop and look at a road map, so that
economics doesn't change as many languages do over time, but remains a
study of economic behavior, and not simply a maximization function.

\section{Proposal}\label{proposal}

\subsection{Abstract}\label{abstract}

The stock market is a great method for any individual, group of
individuals, or entity to participate in the growth of any particular
company. I would like to argue that today's society has turned markets
into a race for a quick buck, more so than ever, with algorithmic
trading. I believe that markets would be better off if investors held
securities for the long-term, along with some market-makers to provide
liquidity, but high-frequency traders have changed our views on the
markets as a whole.

\subsection{Introduction}\label{introduction}

The stock market has been important our nation's economy since it's
initiation in 1817. Even at initiation, people were trying to make a
quick buck. In this essay, I will argue that people's drive to purchase
individual stocks, and invest long-term in individual companies, is
declining. The fault for this goes to high-speed traders. I believe this
takes away from an important purpose of the stock market, which is to
provide funding for companies that need cash outside of debt, and
provide individuals or groups of individuals a chance to be an owner in
hundreds or thousands of companies that they believe in and would like
to see succeed long-term. This purpose has been morphed by algorithmic
traders into a means for quick money, at the cost of investors and
companies.

\subsection{Literature Review}\label{literature-review}

\subsubsection{Buchanan}\label{buchanan-2}

I will use Buchanan in my paper to discuss his theory of a roadmap, and
how economists need to look back and see that they are straying from
their subject matter. I believe that the stock market has strayed from
its original path, and investors are becoming more aggressive at making
money from those with less technology and knowledge rather than
profiting from long-term belief in strong companies.

\subsubsection{Hayek}\label{hayek}

I will use Hayek's paper to discuss how those with a technological
advantage in trading that use that technology for their gain are acting
dishonestly. \citet{Hayek1945}

\subsubsection{Vuorenmaa}\label{vuorenmaa}

I will use this paper to discuss the flash effect that increases
volatility in markets not proven to be caused by, but exaggerated by
algorithmic trading machines. I will discuss predatory trading through
front-frunning, and how this is taking away from the desire of
individuals and other groups to invest. \citet{vuorenmaa2013good}

\subsubsection{Boettke}\label{boettke}

I will use this paper to discuss the importance of investors in society
to understand their investments in companies, and want to see growth in
the individual company, rather than make pennies on the dollar on
multiple trades per second without any sense of how you are helping or
hurting the company whose stock you are trading. This essay speaks of
societal impacts in economic research rather than simply using
quantitative methods. I will use it to argue the societal impact of
algorithmic trading takes the average investor's mindset out of the
original idea of investment. \citet{boettke2013riding}

\section{Outline}\label{outline}

\begin{itemize}
\item
  Introduction
\item
  I will talk about the reasons I believe the stock market started, and
  discuss the social impact that came about by allowing individuals to
  take ownership in companies
\item
  I will talk about how even from the beginning we strayed from this
  concept, and used the stock market to find arbitrage opportunities
\item
  I will discuss the literature I have cited, and talk about how it ties
  into my argument that we have regressed from the original social
  impact that the stock market had on our economy
\item
  I will discuss today's new technology, and how arbitragers and
  high-frequency traders are causing distrust in the stock market as a
  whole for those without technology, and how it is doing more harm than
  good
\item
  Conclusion
\end{itemize}

\newpage
\singlespacing 
\bibliography{./master.bib}

\end{document}
