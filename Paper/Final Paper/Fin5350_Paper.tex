\documentclass[11pt,]{article}
\usepackage[left=1in,top=1in,right=1in,bottom=1in]{geometry}
\newcommand*{\authorfont}{\fontfamily{phv}\selectfont}
\usepackage[]{mathpazo}


  \usepackage[T1]{fontenc}
  \usepackage[utf8]{inputenc}



\usepackage{abstract}
\renewcommand{\abstractname}{}    % clear the title
\renewcommand{\absnamepos}{empty} % originally center

\renewenvironment{abstract}
 {{%
    \setlength{\leftmargin}{0mm}
    \setlength{\rightmargin}{\leftmargin}%
  }%
  \relax}
 {\endlist}

\makeatletter
\def\@maketitle{%
  \newpage
%  \null
%  \vskip 2em%
%  \begin{center}%
  \let \footnote \thanks
    {\fontsize{18}{20}\selectfont\raggedright  \setlength{\parindent}{0pt} \@title \par}%
}
%\fi
\makeatother




\setcounter{secnumdepth}{0}



\title{What Role Should Algorithmic Trading Play?  }



\author{\Large \vspace{0.05in} \newline\normalsize\emph{}  }


\date{}

\usepackage{titlesec}

\titleformat*{\section}{\normalsize\bfseries}
\titleformat*{\subsection}{\normalsize\itshape}
\titleformat*{\subsubsection}{\normalsize\itshape}
\titleformat*{\paragraph}{\normalsize\itshape}
\titleformat*{\subparagraph}{\normalsize\itshape}


\usepackage{natbib}
\bibliographystyle{apsr}



\newtheorem{hypothesis}{Hypothesis}
\usepackage{setspace}

\makeatletter
\@ifpackageloaded{hyperref}{}{%
\ifxetex
  \usepackage[setpagesize=false, % page size defined by xetex
              unicode=false, % unicode breaks when used with xetex
              xetex]{hyperref}
\else
  \usepackage[unicode=true]{hyperref}
\fi
}
\@ifpackageloaded{color}{
    \PassOptionsToPackage{usenames,dvipsnames}{color}
}{%
    \usepackage[usenames,dvipsnames]{color}
}
\makeatother
\hypersetup{breaklinks=true,
            bookmarks=true,
            pdfauthor={ ()},
             pdfkeywords = {Markets, Efficiency, Trading, Algorithmic Trading, High Frequency
Trading},  
            pdftitle={What Role Should Algorithmic Trading Play?},
            colorlinks=true,
            citecolor=blue,
            urlcolor=blue,
            linkcolor=magenta,
            pdfborder={0 0 0}}
\urlstyle{same}  % don't use monospace font for urls



\begin{document}
	
% \pagenumbering{arabic}% resets `page` counter to 1 
%
% \maketitle

{% \usefont{T1}{pnc}{m}{n}
\setlength{\parindent}{0pt}
\thispagestyle{plain}
{\fontsize{18}{20}\selectfont\raggedright 
\maketitle  % title \par  

}

{
   \vskip 13.5pt\relax \normalsize\fontsize{11}{12} 
\textbf{\authorfont } \hskip 15pt \emph{\small }   

}

}







\begin{abstract}

    \hbox{\vrule height .2pt width 39.14pc}

    \vskip 8.5pt % \small 

\noindent Algorithmic trading has seen rapid growth in recent years. Many oppose
algorithmic trading as they argue it provides an unfair advantage to
those with high-speed capabilities through the use of front-running. I
argue in this paper, however, that algorithmic trading increases
liquidity, narrows spreads, makes markets more efficient, and is a net
positive for markets.


\vskip 8.5pt \noindent \emph{Keywords}: Markets, Efficiency, Trading, Algorithmic Trading, High Frequency
Trading \par

    \hbox{\vrule height .2pt width 39.14pc}



\end{abstract}


\vskip 6.5pt

\noindent \doublespacing I would like to open up my discussion of high frequency trading by
referencing a paper by Buchanan. Buchanan begins by disputing a
seemingly simple phrase by Jacob Viner, that `economics is what
economists do, and economists are those who do economics'
\citet{Buchanan1979}. He writes about how the classic form of what an
economics is has been adapted to be methods of resource allocation, but
he disregards this definition, and argues that it should be defined by
trade and exchange.

I am referencing this since trading has all but remained classical.
Trading used to occur all on the trading floor, with traders doing
everything with pen and paper. We have continued to advance and now
almost no trading occurs on the trading floor, but instead occurs within
seconds or even milliseconds. Nowadays speed is the one of the most
valuable assets when it comes to trading.

High frequency traders are defined by the SEC as ``professional traders
acting in a proprietary capacity that engage in strategies that generate
a large number of trades on a daily basis \citet{menkveld2013high}''
High frequency traders actually lose money on their inventory, and
simply attempt to end each trading day as flat as possible on their
actual inventory. They make all of their gains by manipulating the
bid-offer spread. High frequency trading is a subset of algorithmic
trading, which is where traders employe computer algorithms to make
trades on their behalf \citet{zhang2010high}.

The trading world is now in a race to trade at the speed of light. For
reference, the Hibernia Express, a fibre-optic line running from New
York to London's financial centers, was built for \$300 Million. After
this large investment, communication speeds only increased by 2.6
milliseconds, or about 10\% \citet{buchanan2015trading}. This seems like
a steep price to pay for simple milliseconds of speed, but in today's
trading world, this will increase efficiency and cut costs in the long
run. Many trading centers use micro-wave technology to perform trading,
and the future holds many new technologies, such as neutrinos. Neutrinos
travel at the speed of light, and can pass through solid objects
including earth. This is all to increase trading speeds by fractions of
a second.

This paper goes on to discuss one of the reasons people oppose
algorithmic trading. The flash crash that occurred in May 2010, which
was an unexplainable system-wide failure due to unforeseen ways that
different algorithms interacted, is an event that could occur again at
any moment. We continue to develop new technology to become more aware
of how algorithms will respond to different events, but there is always
risk when new technologies are employed, and these can be very
substantive.

This flash crash is described in more detail by
\citet{vuorenmaa2013good}. He then also continues the paper by
discussing the problem of front-running. Front-running is where a
corporation puts in a large purchase of securities, and a high-frequency
trader uses their speed to place a trade in front of the corporation.
After the corporation's trade goes through, the high-speed trader will
sell the security milliseconds after, having made a profit since the
security price will rise due to the large trade volume occuring. This is
a problem that many see with high-speed trading, since it is an unfair
advantage that technology provides.

I would like to argue, however, that the liquidity provided in the
market by high-frequency traders, as well as the tightened spreads, make
the market much more efficient and save investors money, even if
front-running is present. This same paper by Vuorenmaa goes on to talk
about these same facts. It states that daily trading volumes have more
than tripled since high-frequency trading has taken off.

Yet another supporting paper for Algorithmic Trading talks of the
proportion of trades that algorithmic traders have taken
\citet{hendershott2011does}. Here, it mentions in 2009 algorithmic
trading was over 73\% of the daily trading volume in the US. This volume
drastically helps liquidity by allowing individual and institutional
investors to sell securities much quicker, with less of a loss burdened
by large spreads for less liquid securities.

Later it is stated that spreads are tightened through decreased in the
amount of price discovery related to trades. Simply put, prior to
algorithmic trading, much of the change in prices due to news, economic
factors, or other data wasn't reflected in prices until a large trade
was made. With algorithmic trading, however, these events are reflected
much more quickly without the need of these adverse trades. This helps
investors have more faith in market efficiency and be more sure they
will not lose large portions of an investment return through spreads
when trading less liquid securities.

In a recent paper, Hasbrouck added another positive effect algorithmic
trading has on markets \citet{hasbrouck2018high}. He talked about how
price discovery helps negate the possibility of quote stuffing. This is
the term for when an institutional investor places a large order and
cancels, and then repeats this to confuse markets and manipulate stock
prices. With algorithmic trading, quotes are placed much more rapidly
and it is much more difficult to stuff quotes since price discovery
occurs much more efficiently.

This price discovery has done wonders for the efficiency of markets.
Market volatility is by nature unpredictable. The fact that high speed
trading can encorporate new news, whether positive or negative, within
milliseconds is amazing. This helps smooth the effect of unanticipated
news since all of the price changes happen instantaneously, so there
won't be as much of an investor selloff or large purchase as investors
try to get ahead of the news before it reaches others. Now that the news
is incorporated instantaneously through high speed trading, traders are
much less likely to act rashly.

High frequency trading also helps reduce market volatility. With all of
the orders flying in from high speed traders across the globe, there
will be a lot smaller shock should an institutional investor place a
large order for a certain equity. For less trading stocks, without high
frequency traders, when a large order is placed, it is likely the price
shifts upward in a large fashion. High speed traders help significantly
reduce this effect \citet{zhang2010high}.

Volatility is bad for investors because it adds unnecessary risk. If it
is possible for you to buy a stock, only to have the price drop
significantly recently after for no reason other than a large sell order
being placed by an investor with much more capital than you, it is a lot
harder to determine if a stock is worth the purchase. Algorithms used by
high frequency traders search the whole stock market, and can find these
less liquid securities that may not be purchased often by investors, and
will trade them automatically if there is a small margin to be made.
This makes the market much more efficient as a whole.

Going back to the reference of Buchanan, I believe that markets were
made to be efficient. When trading occurred on scraps of paper, the
inherent risks involved were much greater than they are now with the
efficiency introduced by technology. This original idea of trading has
changed so much. Today, most traders don't even trade, but simply
program a computer to trade for them.

Maybe things were meant to evolve with society, and not remain in their
classical sense. Buchanan refutes the idea that economists know what
economics is, but perhaps economics has evolved to not only be about
trade, but also be about utility maximization, and has merged itself
with mathematics. As many other industries have encorporated
mathematics, maybe economics has done the same, and the classical sense
Buchanan defined economics is no longer applicable.

This is definitely the case with classical trading. No one is going to
say an algorithmic trader isn't a trader because they don't do any
trading with pen and paper. Trading has advanced with technology and is
better off for it. High speed trading provides quicker price discover,
higher liquidity, and overall makes the markets more efficient. There
will be many more discoveries in the near future to make trading
quicker, more efficient, and more technologically advanced.

\newpage
\singlespacing 
\bibliography{./master.bib}

\end{document}
